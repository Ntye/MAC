\section{\textbf{\underline{EXERCICES CONCERNANT LE MAC}}}

Cette partie comporte une serie de questions permettant de nous rassurer que vous avez bien compris les MACs. Vous constituerez des équipes de 3 personnes pour y répondre. 

%----------------------Exo 1-----------------------------
\subsection{\textbf{\underline{Exercice 1}}} 
\label{Exo 1}

\begin{enumerate}
    \item [\textbf{Q01}] Laquelle/Lesquelles de ces affirmations est erronnée :
    \begin{itemize}
        \item [i.] Les MACs permettent de garantir l'intégrité des messages
        \item [ii.] Les MACs permettent de garantir la confidentialité des messages
        \item [iii.] Les MACs permettent de garnatir la non répudiation
        \item [iv.] Les MACs permettent de garantir la chronologie des messages
    \end{itemize}
    
    \item [\textbf{Q02}:] Comment fonctionnent les HMACs ?
    
    \item [\textbf{Q03}:] Comment fonctionnent les NMACs ?

    \item [\textbf{Q04}:] Comment fonctionnent les CMACs ?
    
    \item [\textbf{Q05}:] Quelles sont les limites de HMACs ?
    
    \item [\textbf{Q06}:] Quelles sont les limites des CBC-MACs ?
    
    \item [\textbf{Q07}:] Comment faire pour garantir la chronologie des messages quand on utilise les MACs ?
    
    \item [\textbf{Q08}:] Comment renforcer les CBCMACs ?
    
    \item [\textbf{Q09}:] Quels sont les avantages des HMACs ?
    
    \item [\textbf{Q10}:] Comment renforcer les HMACs ?
    
    \item [\textbf{Q11}:] Pour les MACs à chiffrement par blocs, quels sont les critères d'un bon générateur pseudo-aléatoire de clés ?
    
    \item [\textbf{Q12}:] En utilisant uniquement les MACs, peut-on garantir la réception de tous les messages ?
    
    \item [\textbf{Q13}:] Quelles améliorations proposez vous pour les MACs ?\\
\end{enumerate}

\par Pour aller à la \nameref{C_Exo 1}

%----------------------------------------------------------

%----------------------Exo 2-------------------------------
\subsection{\textbf{\underline{Exercice 2}}}
\label{Exo 2}

Soit H une fonction de hachage qui prend en entrée un message m de longueur arbitraire et qui renvoie une sortie de longueur fixe n bits.

On définit le MAC suivant : pour un message m et une clé secrète k de longueur n bits, on calcule MAC(m,k)=H(m$\parallel$k), où $\parallel$ désigne la concaténation de chaînes de bits.
\begin{enumerate}
    \item [\textbf{1.}] Montrer que ce MAC n’est pas sûr, c’est-à-dire qu’un adversaire peut forger un message valide avec une probabilité non négligeable, sans connaître la clé secrète k.
    \item [\textbf{2.}] Proposer une modification du MAC pour le rendre plus sûr, en utilisant la même fonction de hachage H et la même clé secrète k.
\end{enumerate}

\subsubsection{\textbf{\underline{References de cet exercice}}}
Jean-Guillaume Dumas,Jean-louis Roch,Eric Tannier,Sebastien Varette,"Theorie Des Codes" P.185\\

\par Pour aller à la \nameref{C_Exo 2}

%-----------------------------------------------------------

%--------------------Exo 3-----------------------------------
\subsection{\textbf{\underline{Exercice 3}}}
\label{Exo 3}

\textbf{\underline{NB:}}Pour cette exercice, s'il vous plait veuillez ajouter les zeros au debut de la chaine 

Voici le code python d'un générateur pseudo aléatoire :

\begin{Pythoncode}[numbers=left, caption={Python Code}]
def genPseudoAl(ke : str, i : int):
    s = ""
    l = TAILLE // i
    l %= TAILLE
    m = 1
    while len(s) != TAILLE:
        s += ke[l]
        m += 1
        l += TAILLE // m
        l %= TAILLE
    return s
\end{Pythoncode}


Supposons 	qu'une 	clé 	secrète 	soit 	: 	k 	= 
'101010010101001110101010010110101011101001011001010
111010010101010101111001111011101111110110011011010
101001101011001110111100'

Considérons les messages suivants : "Bonjour, comment vas tu", "Salut, à bientôt", "Rendez-vous ce soir à 20h", "J'ai adoré cette sortie", "J'aime coder en python", "Je déteste le langage PHP", "Tu es un ange", "Toi \& moi on est pareils"  sachant que l'algorithme de calcul de MAC convertit les lettres et les chiffres (les caractères spéciaux aussi) en binaire suivant la norme ASCII à 7 chiffres, calaculez le MAC de chaque message (utilisez la convention CBC MAC). 

Vous utiliserez le générateur pour calculer le vecteur d'initialisation d'abord (en mettant i à 1) puis vous l'utiliserez pour trouver chaque vecteur de chiffrement intermédiaire (à chaque nouveau vecteur on incrémente i avant de lancer le générateur).\\

\par Pour aller à la \nameref{C_Exo 3}
%----------------------------------------------------------

%-----------------------Exo 4------------------------------
\subsection{\textbf{\underline{Exercice 4}}}
\label{Exo 4}

Supposons que vous ayez une clé secrète de 8 chiffres (par exemple, 12345678) et un message de texte brut (par exemple, "Hello, world!"). Votre tâche est de générer un MAC en utilisant l'algorithme HMAC-SHA256.

\begin{enumerate}
    \item Convertissez la clé secrète en une séquence de chiffres hexadécimaux (base 16). Par exemple, 12345678 devient 3132333435363738.
    
    \item Concaténez la clé secrète convertie avec le message de texte brut. Par exemple, si le message est "Hello, world!", la concaténation serait 3132333435363738Hello, world!

    \item Utilisez une fonction de hachage SHA256 pour calculer le haché (digest) de la concaténation obtenue.

    \item Le haché obtenu est le MAC généré.\\
\end{enumerate}

\par Pour aller à la \nameref{C_Exo 4}
%---------------------------------------------------------
